\documentclass{article}

\usepackage{graphicx}
\usepackage{libertine}
\usepackage{libertinust1math}

\usepackage{xcolor}

\usepackage{hyperref}
\hypersetup{
    colorlinks=true,
    linkcolor=strongblue,
    filecolor=strongblue,      
    urlcolor=strongblue,
}

\usepackage{pgfpages}
\usepackage{tikz}
\usetikzlibrary{calc}
\usetikzlibrary{matrix}
\usetikzlibrary{positioning}
\usepackage{array}

\usepackage[active,tightpage]{preview}

\pgfdeclarelayer{background}
\pgfsetlayers{background,main}

%%%%%%%%%%%%%%%%%%%%%%%%%%%%%%%
% DOCUMENT
\begin{document}
\begin{preview}


\def \mywebsite{https://mohnjahoney.github.io}


% Colors
%\definecolor{tempborder}{rgb}{0.9, 0, 0.9} % show temp borders for debugging
\definecolor{tempborder}{rgb}{1, 1, 1} % hide temp borders

%\definecolor{strongblue}{rgb}{0.2, 0.3, 0.7} % main theme color
\definecolor{strongblue}{rgb}{0.25, 0.35, 0.75} % main theme color
%\definecolor{strongblue}{rgb}{0.3, 0.4, 0.8} % main theme color


% Set independent lengths
\newlength{\sidebarwidth}
\setlength{\sidebarwidth}{6cm} % sidebar contains a few small boxes of info on the left

\newlength{\headerheight}
\setlength{\headerheight}{4cm} % height of header containing name and contact

\newlength{\divisionlinewidth}
\setlength{\divisionlinewidth}{0.1cm} % width of main "cross" lines that divide the page into 4

\newlength{\divisionlinepad}
\setlength{\divisionlinepad}{0.5cm} % distance from division line for content nodes

\newlength{\sectionpad}
\setlength{\sectionpad}{0.37cm} % distance between content nodes

% This is already defined by tabular
\setlength{\tabcolsep}{0cm} % This allows us to more easily get the final width correct.

\newlength{\innertextpad}
\setlength{\innertextpad}{1.5ex} % distance between (below) each element of descriptive text within a section

\newlength{\workinnertextpad}
\setlength{\workinnertextpad}{0.6ex} % distance between (below) each work experience - squish this more than regular \innertextpad

\newlength{\sidebaritemspacing}
\setlength{\sidebaritemspacing}{1.2ex} % sidebar text spacing - space between elements within each sidebar element

\newlength{\sidebarinnersep}
\setlength{\sidebarinnersep}{1.2ex} % spacing around text within colored node box

\newlength{\splitradius}
\setlength{\splitradius}{1cm} % size of circle centered at the main cross. this is just graphical

% Compute dependent lengths

% TODO: should we splot the divisionlinewidth between left and right equally?
\newlength{\mainbarwidth}
\setlength{\mainbarwidth}{\pdfpagewidth - \sidebarwidth - \divisionlinewidth} % width of main panel on right

\newlength{\bodyheight}
\setlength{\bodyheight}{\pdfpageheight - \headerheight - \divisionlinewidth} % height of main panel (not header)

\newlength{\bulletwidth}
\setlength{\bulletwidth}{3cm} % width of bullets / bold points within main section

\newlength{\descriptionwidth}
\setlength{\descriptionwidth}{\mainbarwidth - \bulletwidth - \divisionlinepad} % width of main descriptive text within main section (after bullets)

\newlength{\vennrad}
\setlength{\vennrad}{\headerheight * \real{0.5}} % size of venn diagram circle in upper left corner
% NOTE: that \real{} has to follow the \headerheight - ugh

%%%%%%%%%%%%%%%%%%%%%%%%%%%%%%%
%%% TIKZPICTURE
\begin{tikzpicture} [
% Layout stuff
mat/.style={draw=tempborder, rectangle, matrix of nodes, rounded corners, line width=0.1pt, inner sep=0pt, outer sep=0pt, nodes={anchor=center}}, % general matrix of node features
sec_title_mat/.style={mat, fill=white}, % Use "matrix of nodes" for section title since it seems the easiest way to get the nice circles next to the text.
sidebar_sec_title_mat/.style={mat, draw=none, fill=white, align=right}, % 
profile_node/.style={draw=none, anchor=south west, inner sep=0pt, font=\sffamily, text width=\mainbarwidth - 4*2*\divisionlinepad, align=center}, %
mainbar_body_node/.style={draw=none, anchor=north west, inner sep=0pt, font=\sffamily}, %
sidebar_body_node/.style={draw=tempborder, rounded corners, fill=strongblue!8!white, inner sep=\sidebarinnersep, outer sep=0}, %
% Other stuff
sec_title/.style={font=\Large, anchor=center}, % section title
sec_title_circle/.style={draw=white, circle, line width=2pt, minimum width=0.4cm, anchor=center}, % circles used to demarkate section titles
sidebar_sec_title_circle/.style={draw=white, circle, line width=2pt, minimum width=0.4cm, anchor=center}, % circles that mark section headers
remember picture]

%% draw image
%\node[inner sep=0, opacity=0.3] at (current page)
%{\includegraphics[width=\paperwidth,height=\paperheight]{me.png}};

%%% Split point - this is the main organizational coordinate
\node[draw=strongblue, circle, line width=2pt, minimum width=\splitradius, inner sep=0] (split) at (\sidebarwidth, \pdfpageheight - \headerheight) {};

% Draw big organizational lines
\draw[strongblue, line width=\divisionlinewidth] (split.center) -- ++(\mainbarwidth, 0);
\draw[strongblue, line width=\divisionlinewidth] (split.center) -- ++(-\sidebarwidth, 0);
\draw[strongblue, line width=\divisionlinewidth] (split.center) -- ++(0, \headerheight);
\draw[strongblue, line width=\divisionlinewidth] (split.center) -- ++(0, -\bodyheight);

%%% Corner Icon

% Intersection area - "football"
\begin{scope}
\clip ($(split.center) + (-\vennrad - \divisionlinewidth/2, \vennrad + \divisionlinewidth/2)$) circle (\vennrad);
\draw[draw=none, line width=0, fill=yellow, opacity=1.0] (split.center) circle (1.0*\vennrad);
\end{scope}

% Foliation
\begin{scope}
\clip ($(split.center) + (-\vennrad - \divisionlinewidth/2, \vennrad + \divisionlinewidth/2)$) circle (\vennrad);

% Draw a sequence of overlapping regions with transparency to make a foliation.
\foreach \up in {7, 6, 5, 4, 3, 2, 1}{
\filldraw[draw=black!30, line width=1pt, fill=green, fill opacity=0.1] ($(split.center) + (-\vennrad, 0)$) -- (split.center) -- ++(\up*12+90:\vennrad) -- ++(-2*\vennrad, 0) -- ++(0, -2*\vennrad) -- cycle;

% TODO: could use the control points to make foliation curved
%\filldraw[draw=black!60, semithick, fill=blue, fill opacity=0.1] ($(split.center) + (-\vennrad, \up)$) .. controls ++(-2,-2+\up) and (-1,0+\up) ..  (2,-1+\up) -- (3,0) -- (3,-5) -- (-3,-5) --   cycle;
}

% Add cool tapered curve to show the upper boundary
\draw[draw=black!20, line width=6pt] ($(split.center) + (-2*\vennrad + \divisionlinewidth/2, \vennrad)$) arc (180:0:\vennrad);

\end{scope}


%%% Profile
\node[profile_node, font=\large, anchor=south] (profile) at ($(split.center) + (\mainbarwidth/2, 1*\sectionpad)$) {
\emph{
Experienced and inventive physicist, applied-math researcher, data scientist, and educator.
My interests include time-series, information theory, reacting flows, and health informatics.
}\\
%I'm a rad dad who endeavors to make the world a fizzier place. 
%A physicist by training and jazz saxophonist by night, I approach the world with both an analytic mind and a the desire for a deep pocket.
};


%%% Name
\node[color=white, fill=strongblue, draw=none, line width=2pt, font=\huge, scale=1.8, rounded corners, anchor=south] (name) at ($(profile.north) + (0, 1*\sectionpad)$) {\textcolor{white}{~~~~John R. Mahoney~~~~}};


%%% Communication Skills
\matrix[sec_title_mat, anchor=north west,
row 1/.style={sec_title}] (communication_skills_title) at ($(split.center) + (\divisionlinepad, -2*\sectionpad)$) {
\node[sec_title_circle, draw=
%strongblue,
black!20,
fill=green, fill opacity=0.5] {};&
~~COMMUNICATION~~&
\node[sec_title_circle] {};\\
};

\node[mainbar_body_node] (communication_skills) at ($(communication_skills_title.south west) + (0.0, -\sectionpad)$) {
\begin{tabular} {>{\bfseries\normalsize}p{\bulletwidth} p{\descriptionwidth}}
Written: &  Wrote and co-authored over 25 papers published in high-impact physics journals, leading to advancement of theory in: \href{https://mohnjahoney.github.io/pdfs/Times_Barbed_Arrow.pdf}{time-series prediction}, \href{https://mohnjahoney.github.io/pdfs/A_Turnstile_Mechanism_For_Fronts_Propagating_In_Fluid_Flows.pdf}{reacting fluid flows}, and \href{https://mohnjahoney.github.io/pdfs/Occams_Quantum_Strop.pdf}{quantum resource theory}; Important in securing co-authored grants from NSF (\$350k) and Templeton Foundation (\$440k).
\\[\innertextpad]
%
Verbal: & Designed and delivered over 35 research presentations 
%on time-series analysis, reacting fluid flows, and quantum information 
in venues such as Singapore, Amsterdam, Paris, Budapest, and Sendai.
Awarded \href{https://mohnjahoney.github.io/pdfs/posters/Oberwolfach2014.pdf}{best poster} at ``Mixing, Transport, and Coherent Structures Workshop'' at the \href{https://www.mfo.de/}{MFO}.
Our theories have been applied in dozens of other theoretical and experimental works.
Taught, in lecture and small-group setting, over 1000 students.
% TODO: add "Listen to: link to a talk I've given."
\\[\innertextpad]
Visual: & Value clarity, simplicity and aesthetics in communication. New \href{https://mohnjahoney.github.io/images/burning_front_topology.png}{reacting flow diagram} advances the canonical idea of phase portrait; applicable for heat transport and engine design. Promoted use of \href{https://mohnjahoney.github.io/images/FoliationCryptic.pdf}{Venn diagrams} for information theory; discovered new concepts of randomness and structure in time-series; impacted limits on algorithm design and computational architecture. Distilled complex relationships between sleep and blood sugar into rich and digestible \href{https://mohnjahoney.github.io/images/mosaic_nosharey.pdf}{graphic}.
\\
\end{tabular}
};


%%% Analytical Skills
\matrix[sec_title_mat, anchor=north west,
row 1/.style={sec_title}] (analytical_skills_title) at ($(communication_skills.south west) + (0.0cm, -2*\sectionpad)$) {
\node[sec_title_circle, draw=black!20, fill=green, fill opacity=0.5] {};&
~~ANALYTICAL SKILLS~~&
\node[sec_title_circle] {};\\
};

\node[mainbar_body_node] (analytical_skills) at ($(analytical_skills_title.south west) + (0.0, -\sectionpad)$) {
\begin{tabular} {>{\bfseries\normalsize}p{\bulletwidth} p{\descriptionwidth}}
Research: & Ability to assimilate and utilize knowledge from new areas. In my work on reacting fluids, I connected to a number of existing fields: invariant manifolds, FT Lyapunov exponents, ARD equation, catastrophe theory, vehicle path planning, differential geometry.\\[\innertextpad]
Critical Thinking: & Sharp eye for details and definitions. Reframed an assumption in the literature to build a fruitful research avenue studying \emph{crypticity} and \emph{cryptic order}. \\[\innertextpad]
Data and Code & Skilled scientific coder. Created Python pipeline for data on diabetes patients: clean, process, analyze (multiple pair lagged regression), visualize. \\
\end{tabular}
};


%%% Work Experience
\matrix[sec_title_mat, anchor=north west,
row 1/.style={sec_title}] (work_experience_title) at ($(analytical_skills.south west) + (0.0cm, -2*\sectionpad)$) {
\node[sec_title_circle, draw=black!20, fill=green, fill opacity=0.5] {};&
~~WORK EXPERIENCE~~&
\node[sec_title_circle] {};\\
};

\node[mainbar_body_node] (work_experience) at ($(work_experience_title.south west) + (0.0, -\sectionpad)$) {
\begin{tabular} {>{\bfseries\normalsize}p{\bulletwidth} p{\descriptionwidth}}
Fall 2020&Math Specialist: UC Davis\\[\workinnertextpad]
Summer 2020&Course Designer and Instructor: UC Davis\\[\workinnertextpad]
Oct 2019&Math Lecturer: Napa Valley College\\[\workinnertextpad]
Spring 2019&Physics Lecturer: UC Davis\\[\workinnertextpad]
Fall 2018&Math Lecturer: CSU Maritime\\[\workinnertextpad]
2017-2018&Consultant: Dept. Biomedical Informatics, Columbia University\\[\workinnertextpad]
2017-2018&Math Lecturer: UC Davis\\[\workinnertextpad]
2015-2017&Project Scientist: UC Davis\\[\workinnertextpad]
2010-2015&Postdoctoral Scholar: UC Merced\\
\end{tabular}
};


%%% Education
\matrix[sec_title_mat, anchor=north west,
row 1/.style={sec_title}] (education_title) at ($(work_experience.south west) + (0.0cm, -2*\sectionpad)$) {
\node[sec_title_circle, draw=black!20, fill=green, fill opacity=0.5] {};&
~~EDUCATION~~&
\node[sec_title_circle] {};\\
};

\node[mainbar_body_node] (education) at ($(education_title.south west) + (0.0, -\sectionpad)$) {
\begin{tabular} {p{\descriptionwidth}}
Ph.D. in Physics, UC Davis, \emph{Extensions of the Theory of Computational Mechanics}, advisor: James P. Crutchfield\\[\innertextpad]

B.S. in Physics and Mathematics, CSU Chico\\[\innertextpad]
Williams College for Physics, Mathematics and Music\\[\innertextpad]
\end{tabular}
};


%%%%%%%%%%%
% Sidebar
%%%%%%%%%%%

\matrix[sidebar_sec_title_mat, anchor=north east,
row 1/.style={sec_title}] (interpersonal_skills_title) at ($(split.center) + (-\divisionlinepad, -2*\sectionpad)$) {
\node[sidebar_sec_title_circle] {};&
INTERPERSONAL&
\node[sidebar_sec_title_circle] {};\\
};

\node[sidebar_body_node, anchor=north east, align=right, font=\large, text width=\sidebarwidth - \divisionlinepad - 2\sidebarinnersep] (interpersonal_skills) at ($(interpersonal_skills_title.south east) + (0.0, -\sectionpad)$) {
Excellent listener\\[\sidebaritemspacing]
Flexible team-player\\[\sidebaritemspacing]
%Work well in small teams\\[\sidebaritemspacing]
Effective mentor\\%: (3 PhD, 7 undergrads)\\
};


%%% Coding Projects
\matrix[sidebar_sec_title_mat, anchor=north east,
row 1/.style={sec_title}] (coding_projects_title) at ($(interpersonal_skills.south east) + (0.0cm, -2*\sectionpad)$) {
\node[sidebar_sec_title_circle] {};&
~~CODING PROJECTS~~&
\node[sidebar_sec_title_circle] {};\\
};

\node[sidebar_body_node, anchor=north east,  align=right, font=\large, text width=\sidebarwidth - \divisionlinepad - 2\sidebarinnersep] (coding_projects) at ($(coding_projects_title.south east) + (0.0, -\sectionpad)$) {
\href{https://github.com/mohnjahoney/magnacules}{Python \& Physics course}\\[\sidebaritemspacing]
\href{link}{Burning Invariant Manifolds}\\[\sidebaritemspacing]
\href{link}{CMPy} contributor\\[\sidebaritemspacing]
\href{https://github.com/mohnjahoney/simpsons_paradox}{Simpson's Paradox}\\[\sidebaritemspacing]
\href{link}{timesquare}\\[\sidebaritemspacing]
\href{https://github.com/mohnjahoney/resume}{r\'esum\'e template}\\
};


%%% Programming Skills
\matrix[sidebar_sec_title_mat, anchor=north east,
row 1/.style={sec_title}] (programming_skills_title) at ($(coding_projects.south east) + (0.0cm, -2*\sectionpad)$) {
\node[sidebar_sec_title_circle] {};&
~~PROGRAMMING~~&
\node[sidebar_sec_title_circle] {};\\
};

\node[sidebar_body_node, anchor=north east,  align=right, font=\large, text width=\sidebarwidth - \divisionlinepad - 2\sidebarinnersep] (programming_skills) at ($(programming_skills_title.south east) + (0.0, -\sectionpad)$) {
Python: np, sp, mpl, pd\\[\sidebaritemspacing]
GUI / interactive\\[\sidebaritemspacing]
git, \LaTeX, beamer, tikz\\[\sidebaritemspacing]
ipython, Jupyter, VS Code\\[\sidebaritemspacing]
MATLAB\\[\sidebaritemspacing]
Mac OS, UNIX\\
};


%%% INTERESTS
\matrix[sidebar_sec_title_mat, anchor=north east,
row 1/.style={sec_title}] (interests_title) at ($(programming_skills.south east) + (0.0cm, -2*\sectionpad)$) {
\node[sidebar_sec_title_circle] {};&
~~INTERESTS~~&
\node[sidebar_sec_title_circle] {};\\
};

\node[sidebar_body_node, anchor=north east,  align=right, font=\large, text width=\sidebarwidth - \divisionlinepad - 2\sidebarinnersep] (interests) at ($(interests_title.south east) + (0.0, -\sectionpad)$) {
Jazz saxophone and piano\\[\sidebaritemspacing]
Soccer, tennis, and hiking\\[\sidebaritemspacing]
Cooking delicious food!\\
};

%%% CONTACT
\node[sidebar_body_node, anchor=north east,  align=center, font=\large, text width=\sidebarwidth - \divisionlinepad - 2\sidebarinnersep, fill=none] (contact) at ($(interests.south east) + (0.0, -3*\sectionpad)$) {
mohnjahoney@gmail.com\\[\sidebaritemspacing]
(530) 601-0524\\[\sidebaritemspacing]
\href{\mywebsite}{mohnjahoney.github.io}\\[\sidebaritemspacing]
\href{https://github.com/mohnjahoney}{\includegraphics[width=0.8cm]{github.pdf}}
\quad
\href{https://www.linkedin.com/in/johnmahoney3}{\includegraphics[width=0.8cm]{LI-In-Bug.png}}
};


\end{tikzpicture}
\end{preview}
\end{document}
